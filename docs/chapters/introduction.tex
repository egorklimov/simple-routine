\chapter{Введение}
\section{Цели}
Данный документ описывает приложение <<\vProjectName>> --- инструмент для ежедневнего планирования. Основные цели приложения --- помочь пользователю отслеживать выполнение рутинных задач, а также способствовать достижению долгосрочных целей.

\section{Соглашения о терминах}
\begin{itemize}
    \item \textbf{Цель} --- масштабная задача, имеющая срок выполнения и план действий к ее достижению.
    \item \textbf{Список задач} --- план необходимых действий для достижения \textbf{цели}.
    \item \textbf{Задача} --- атомарное действие, составляющее \textbf{список задач}.
    \item \textbf{Календарь} --- инструмент для составления персонального расписания выполнения \textbf{задач} внутри приложения.
    \item \textbf{Помодорро} --- таймер длительностью на усмотрение пользователя (по умолчанию --- 25 минут) для сконцентрированной работы над \textbf{задачей}.
    \item \textbf{Шаблон} --- автоматическое создание в \textbf{календаре} заданного пользователем количества копий \textbf{задач} для удобного соблюдения их регулярного выполнения.
\end{itemize}


\section{Предполагаемая аудитория}
Данный документ предназначен для команды разработки проекта <<\vProjectName>>, а именно для разработчиков, менеджеров проекта, тестировщиков, а также для будущих пользователей приложения. Общее видение продукта и основные характеристики представлены в разделе \ref{ch:description} и будут интересны потенциальным пользователям. В разделах \ref{ch:functionality} и \ref{ch:nonfunctional} представлены функциональные и нефункциональные требования, интересные разработчикам и тестировщикам продукта, а также прочие требования, которые могут заинтересовать менеджеров проекта. 

\section{Масштаб проекта}
Изначальная минимальная версия приложения предназначена для русскоязычных пользователей. В дальнейшем возможно расширение продукта и его вывод на международный рынок.

\section{Ссылки на источники}
https://github.com/egorklimov/simple-routine