\chapter{Общее описание}
\label{ch:description}
\section{Видение продукта}
<<\vProjectName>>  --- новый продукт, объединяющий в себе характеристики многих полезных инструментов повседневного использования. Существуют удобные инструменты для отдельных компонентов этого приложения: календарь (Google календарь), заметки (Evernote), трекер привычек (Productive), отслеживане достижения целей (Trello), список задач (Todoist). Однако на данный момент нет приложения, удобным образом объединяющего в себе все необходимые инструменты. simple-routine задуман именно таким.


\section{Функциональность продукта}
\begin{itemize}
    \item ежедневный трекер привычек
    \item календарь с напоминаниями о созданных событиях
    \item доска канбан для выполнения длительных и/или многоступенчатых задач
    \item таймер продуктивности <<помодорро>>
    \item анализ эффективности выполнения задач и следования планам
    \item возможность делиться результатами с другими пользователями
\end{itemize}


\section{Классы и характеристики пользователей}
Приложение подойдет для людей разных возрастных групп. Оно будет полезно для тех, кто хочет успевать больше, кто не хочет/не может держать все цели и задачи в голове, а хочет зафиксировать их в удобном месте и планомерно двигаться к их достижению. Шаблоны будут полезны для тех, кто хочет выработать определенные привычки, а таймер <<помодорро>> --- для тех, кому сложно сосредоточиться на рутинных или больших задачах. Проект планируется интернациональным, с первоначальным релизом на русскоязычную аудиторию.

\section{Среда функционирования продукта}
\label{sec:environment}

simple-routine --- облачный веб-сервис, разрабатываемый с использованием технологии Progressive Web App (PWA), для поддержания различных платформ (ПК/мобильное приложение/адаптивный веб-сайт). Приложение должно легко адаптироваться к росту нагрузки, а также не должно зависеть от среды выполнения, для этого выбран cloud native подход, в разработке необходимо поддерживать возможность горизонтального масштабирования. Сервис должен корректно работать в условиях ухудшения и полной потери связи, для поддержания <<офлайн режима>> планируется использовать механизм service worker.

\section{Рамки, ограничения, правила и стандарты}
\begin{itemize}
    \item Приложение ориентировано на персональное и командное использование. Таким образом, приложение не предполагает доступ к информации неограниченного круга лиц, поэтому не является СМИ согласно Закону РФ от 27.12.1991 N 2124-1 (ред. от 30.12.2020) "О средствах массовой информации" (с изм. и доп., вступ. в силу с 01.01.2021. Перед интернационализацией сервиса необходимо оценить требования регуляторов других стран. Опицонально, можно поддержать внутреннюю социальную сеть, однако, эта возможность не входит в список минимальных требований для запуска продукта и требует дополнительного анализа ограничений, т.к. в таком случае сервис может быть приравнен к СМИ.
    \item К правилам относится неразглашение личной информации пользователя, его целей и событий календаря. (General Data Protection Regulation) 
    \item Сервис должен быть доступен людям с ограниченными возможностями (ГОСТ Р 52872-2019, Web Content Accessibility Guidelines)
\end{itemize}
