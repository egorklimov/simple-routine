\chapter{Функциональность системы}
\label{ch:functionality}

К обязательным функциям продукта относятся следующие функции:
\begin{itemize}
    \item ежедневный трекер привычек
    \item календарь с напоминаниями о созданных событиях
    \item доска канбан для выполнения длительных и/или многоступенчатых задач
\end{itemize}

\section{Ежедневный трекер привычек}
Подойдет для выработки привычки. Приложение даст напоминание выполнить ежедневную запланированную деятельность в указанное время.

\subsection{Описание и приоритет}
Пользователь создает ежедневную задачу, которую хочет выполнять регулярно в течение заданного промежутка времени, чтобы выработать привычку. Приложение уведомляет пользователя и подбадривает на пути к достижению цели

\subsection{Функциональные требования}
Создание/удаление задачи, настройка уведомлений

\section{Календарь событий}

\subsection{Описание и приоритет}
Календарь, в который можно добавлять события и получать уведомления об их приближении

\subsection{Функциональные требования}
Добавление, удаление и редактирование событий, настройка уведомлений на них.

\section{Доска кабан}
Интерфейс, позволяющий визуализировать и структурировать работу в команде над крупными проектами.

\subsection{Описание и приоритет}
Представляет из себя доску, поделенную на этапы выполнения задач, помогает отслеживать ход их выполнения и контролировать процесс крупной задачи

\subsection{Функциональные требования}
Добавление задач, перемещение их по доске, подсчитанные статистики количества выполненной работы.

\section{Таймер продуктвиности <<помодоро>>}
Популярная методика для повышения продуктивности работы

\subsection{Описание и приоритет}
Установлено, что специальным образом установленные промежутки работы и отдыха, позволяют увеличить производительность работы. Таким образом таймер поможет пользователю соблюдать этот режим.

\subsection{Функциональные требования}
Настройка таймера, удобство его использования

\section{Возможность делиться результатами}

\subsection{Описание и приоритет}
Все выполняемые цели могут быть репостнуты в популярные соц. сети

\subsection{Функциональные требования}
Возможность расшарить статус выполенного задания в сеть.