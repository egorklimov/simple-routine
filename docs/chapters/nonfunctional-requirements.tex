\chapter{Нефункциональные требования}
\label{ch:nonfunctional}

\section{Требования к окружению}
Среда функционирования ПО описана в разделе \ref{sec:environment}.\\ <<simple-routine>> - \href{https://github.com/cncf/toc/blob/main/DEFINITION.md#\%D1\%80\%D1\%83\%D1\%81\%D1\%81\%D0\%BA\%D0\%B8\%D0\%B9}{\color{blue}{cloud native}} приложение и должно соответствовать стандартам Cloud Native Computing Foundation. 

Оркестрация должна быть реализована с использованием Kubernetes. Для поддержания высокого уровня доступности сервиса планируется воспользоваться услугами внешних провайдеров, таких как Amazon Web Services и Google Cloud, которые гарантируют высокий уровень качества предоставления услуг в своих соглашениях об уровне услуг (Service Level Agreement, SLA). Гарантированная доступность сервиса <<simple-routine>> должна соответствовать - 95\% в месяц. Для повышения уровня доверия к сервису необходимо сформировать SLA и предоставить его пользователям на основной странице сервиса. 

Необходим режим с пониженным использованием интернет-трафика --- для экономии интернет-трафика пользователя, в таком режиме пользователь должен иметь возможность увидеть объем потребляемого трафика, а также изменить качество вспомогательных элементов (картинки, стили).

\section{Требования к производительности}
Приложение должно эффективно работать на смартфоне, а также на ПК. Страницы web-приложения должны загружаться не дольше 3х секунд. Отклик для мобильного приложения должен составлять не больше секунды. Среднее значение статистик при прохождении анализатора \href{https://developers.google.com/web/tools/lighthouse}{\color{blue}{Google Lighthouse}} должно быть не меньше 95.


\section{Требования к сохранности (данных)}
\begin{itemize}
    \item Процесс аутентификации должен соответствовать стандарту OAuth 2.0, с поддержкой формата токена JSON Web Token (JWT);
    \item Чувствительные пользовательские данные должны должны храниться в зашифрованном виде;
    \item Необходимо гарантировать восстановление пользовательских данных за последние 8 часов работы сервиса при условии падения сервиса;
    \item Способность к резервному копированию данных;
    \item Архивирование действий пользователей и системных событий;
    \item Данные внесенные в <<офлайн режиме>> должны быть синхронизированы с хранилищем при восстановлении связи.
\end{itemize}

\section{Критерии качества ПО}
\begin{itemize}
    \item Возможность установки через GooglePlay с устройства, а также через ПК;
    \item Отсутствие утечек памяти;
    \item Низкий уровень потребления энергии при работе в фоновом режиме;
    \item Выполнение описанных в документе функциональных и нефункциональных требований;
    \item Уровень тестового прокрытия - 85\%;
    \item Прохождение статических анализаторов кода;
    \item Высокий уровень мониторинга работы сервиса и поддержание трассировки пользовательских запросов в серсиве (с использованием стандарта OpenTelemetry).
\end{itemize}


\section{Требования к безопасности системы}
Приложение должно поддерживать шифрование персональных данных пользователей при записи в память во избежание утечки информации. Необходима строгая аутентификация и авторизация, а также привязка процессов внутри приложения к идентификатору пользователя. В случае нарушения безопасности системы необходимо иметь возможность точно оценить уровень утечки и оперативно уведомить пользователей. 


\section{Прочие требования}

Все основые требования описаны в разделах \ref{ch:functionality} и \ref{ch:nonfunctional}.